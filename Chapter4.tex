\chapter{Proposed System}

This chapter will present our proposed system as the solution. We will define our system and show how it solves the problem of maintaining privacy for the users. Then we will discuss about a generic prototype we made using the design from the proposed system. We will describe different parts of the system and how the end user perceives it.
\section{User Identification}
In order to solve the problem we first look at how the users are identified in the system. There are 2 ways by which a bank can identify the actual users
\begin{itemize}
	\item From the logs i.e. transaction data
	
The bank store all the logs or transaction data with real identity of the user. It is easy to access this data by the bank for a given user and extract all of his data.
	\item From a database in the bank system where personal details of the user are stored.	

The bank stores personal data about all its users in a database. As this database lies at the bank, it is possible for the bank to use the database to get the personal information about a user.
\end{itemize}
We can remove this identification in the following ways:
\begin{itemize}
	\item Remove the user identity from the logs and transaction data

Like this there is no way for the bank to get transaction data for a given user as there will be no user identity linked to the logs or transactions.
	\item Either remove the user's personal data or limit the access to this personal database.

If the bank removes the personal database of the users from its side, then there is no way the bank can get personal information of the users.

If the bank limits the access to such database and doesn't really use it for day to day banking purposes then it is also possible for the bank to limit access to the user's personal information.
\end{itemize}
\section{Pseudonym System}
In order to provide privacy to the users, we suggest a new pseudonym system as shown in figure \ref{fig:Pseudonym}.
\begin{figure}[h]
\centering
\includegraphics[width=\textwidth]{figures/Pseudonym}
\caption{Pseudonym Banking System}
\label{fig:Pseudonym}
\end{figure}
In our system we modify the authentication service and replace it with an \textit{Identity Mapping System (IMS)}
\begin{itemize}
	\item Instead of giving the \textit{User ID} to the bank we actually replace it with a \textit{pseudonym}.
	\item The IMS sends the \textit{policy} as well as the \textit{account ID} to the bank.	
	\item The bank then uses this information to provide services to the customer.
	\item The bank doesn’t need to store the mapping databases.
\end{itemize}
The IMS can either be in the bank, in a separate department,or it can be managed by a 3rd party. This way the bank doesn’t need to know the exact identity of the user to provide them with the services.
Also, the bank can still store personal details of the user in case of the legal requirement, but it can be stored at a separate place as it is not needed for day to day operation of the bank.
\section{Properties}
The privacy properties mentioned below are desirable in our pseudonym system:
\begin{itemize}
	\item \textbf{Unlinkability :} If the same pseudonym is used with different identities, or different pseudonyms are used for the same identity, it should not be possible to link these different transactions to the same person. 
	\item \textbf{Partial Information Disclosure :} The information given by a user should be minimum and he should be able to choose what information values he actually wants to be made available to the bank. 
	\item \textbf{Conditional Anonymity Removal :} In case of some discrepancy or legal requirement, the authorities should be able to come in and identify the real user from the Pseudonym.
	\item \textbf{Revocation :} It should be easy to revoke any user. Also, it should be easy to check whether a certain user is revoked or not. 
\end{itemize}
\section{User Privacy}
As in our system the bank never gets the real identity of the user, the user's anonymity to the bank is maintained. Also, the bank doesn't need to store the policies for the users as its all coming from the IMS. As a result, it decreases a lot of load from the bank to store such data. The IMS service adds a layer of pseudonymity in the system. 

\section{Generic Prototype}
We will now talk about a \textit{generic prototype} based on our proposed system. For the end user it doesn't change anything. The end user authenticates to the IMS and then the IMS creates a pseudonym for the user. This pseudonym is then used by the bank to provide services to the end user.
\subsection{Authentication}
\begin{figure}[h]
	\centering
	\includegraphics[width=\textwidth]{figures/Login}
	\caption{User Login Page}
	\label{fig:Login}
\end{figure}
\begin{figure}[h]
	\centering
	\includegraphics[width=\textwidth]{figures/Logged}
	\caption{Authenticated User}
	\label{fig:Logged}
\end{figure}	
User authentication happens as follows:
\begin{itemize}
	\item The user goes to the bank login page.
	\item The user puts his credentials in the login system.
	\item The user's credentials are verified by the IMS and then he is redirected to his account page.
\end{itemize}
One thing to note here is that for a traditional user, this doesn't change anything on the user's end. The user still uses the same process to get access to his account.

As we can see in \ref{fig:Logged}, after authentication, the user is logged in with a pseudonym.	
\subsection{User Information}
\begin{figure}[h]
	\centering
	\includegraphics[width=\textwidth]{figures/Account1}
	\caption{User Information - Session 1}
	\label{fig:Account1}
\end{figure}	
\begin{figure}[h]
	\centering
	\includegraphics[width=\textwidth]{figures/Account1}
	\caption{User Information - Session 2}
	\label{fig:Account2}
\end{figure}
When we go to the user information page we can see the information that is given to the bank by the IMS for a given user.

In our system it is:
\begin{itemize}
	\item Pseudonym
	\item Account ID
	\item Balance
	\item Account Type
	\item Policies
	\begin{itemize}
		\item Withdraw Permission
		\item Transfer Permission
	\end{itemize}
\end{itemize}
As we can see from the figure \ref{fig:Account1} and the figure \ref{fig:Account2}, two different sessions of the same user are logged in with different pseudonyms. The bank has no way to find out that it is the same user who has logged into different sessions.

\subsection{Transactions}
\begin{figure}
	\centering
	\includegraphics[width=\textwidth]{figures/New}
	\caption{New Transactions}
	\label{fig:New}
\end{figure}
In our prototype the user is allowed to do two types of transactions:
\begin{itemize}
	\item Debit
	\item Credit
\end{itemize}
All the transactions that are done by the user are saved with the pseudonym. This pseudonym is the same with which the user has been logged in to the system.

Figure \ref{fig:New} shows the \textit{new transactions} page in the system where the user is allowed to do the transactions.
\begin{figure}[h]
	\centering
	\includegraphics[width=\textwidth]{figures/Transactions}
	\caption{Account Transactions}
	\label{fig:Transactions}
\end{figure}

Figure \ref{fig:Transactions} shows all the transactions that have been done on the given account by the users. As we can see, all the transactions are saved with the pseudonyms of the users.
\begin{figure}[h]
	\centering
	\includegraphics[width=\textwidth]{figures/Download}
	\caption{Download Transactions Option}
	\label{fig:Download}
\end{figure}
\begin{figure}[h]
	\centering
	\includegraphics[width=\textwidth]{figures/File}
	\caption{Downloaded Transactions File}
	\label{fig:File}
\end{figure}	
Our system also allows the users to download the transactions from the \textit{download transactions} page as shown in figure \ref{fig:Download}. These transactions are stored in a csv file. This csv file can be opened by the user as can be seen in the figure \ref{fig:File}.
\FloatBarrier
In this chapter we presented our pseudonym system. Also we have given some certain privacy properties that our pseudonym system should be able to fulfill. Our generic system works with pseudonyms and in the next chapter we will talk about how we can replace our Identity Mapping System with OpenID and IDEMIX.

We have decided to analyze two different systems for our IMS:
\begin{itemize}
	\item OpenID
	\item IDEMIX
\end{itemize}.