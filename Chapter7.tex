\chapter{Evaluation}
This chapter will present the evaluation criteria for our system. After presenting the criteria, we will evaluate the two systems presented in chapters 6 and 7.
\section{Evaluation Criteria}
Below are the evaluation criteria and project goals that we have setup for our system:
\subsection{Unlinkability}
We want to unlink the real identity of the user from the transactions. A user should not have to give his real identity to the bank in order to get the services. Also two different sessions of the same user should be unlinkable i.e. it should not be possible to find out that two sessions are from the same user or two different users.

Being able to link the real identity of the user with the transactions creates a lot of problems. It is possible for someone, who has access to such data, to learn about the financial life of the given individual. So this property is desired to avoid such problems.
\subsection{Escrow}
It should be possible for the authorities to get the real identity of the user in case of legal requirements or discrepancies. But, still the bank should not be able to get the real identities of the users.

Providing anonymity is good but sometimes people take advantage of anonymity on internet. e.g. they might perform some illegal transactions at the bank, while they are anonymous. So the escrow property is required to handle the cases.
\subsection{Minimal Technical Requirements}
It should be easy to be a customer at the bank. The user should not have to change a lot of systems on their side to be a customer. It should also be easier for existing customers to continue using the services of the bank.

In the end, its all about customer. Customers want security but they don't want to sacrifice ease of use. It would be difficult for the bank to keep the customers or to get new customers if it means that they have to invest heavily in IT infrastructure just to be a customer with them. So this requirement is needed to make it easy for customers to get the services from the bank.
\subsection{User awareness}
The user should be aware of the data they are sharing with the bank. The bank should have user consent before storing any data from the user.

People are getting more and more aware of their privacy. They want to know what personal data is being stored by the service providers in order to provide them with the services. This requirement takes care of the case where the user knows exactly what data he is sharing with the bank.
\subsection{Protection of Data}
All the sensitive data about the user should be kept protected. Also, all the bank related sensitive data e.g. account information, policies should remain secure.

Keeping the data protected is a big challenge. If the bank is not able to protect the user's data it gives them bad reputation in the market. Sensitive business data is also valuable to the bank. The bank doesn't want to give this data to anyone else.

\section{OpenID based solution}
Now we will evaluate our OpenID based solution as described in chapter 5. 
\begin{enumerate}
	\item{\textbf{Unlinkability}}
	In the OpenID based system all the transactions are logged using a pseudonym. Hence the user's real identity is unlinked from the transactions. If the same user logs in again, he is given a different pseudonym, hence its not possible to relate with sessions with each other.
	\item{\textbf{Escrow :}}
	In case the authorities need to get the real identity behind a transaction there is an escrow capability. The IMS system stores the mapping database from the pseudonym to the real identity of the user. The authorities need to go to the bank to get the transactions and then to the IMS provider to get the mapping data.
	\item{\textbf{Minimal Technical Requirements :}}
	For the end users, there aren't many technical requirements. They don't have to setup a special hardware to be a customer at the bank. All the interface is web based and can be used in any normal web browser. 
	\item{\textbf{User Awareness :}}
	In the OpenID based system, the user is told during authentication what data about him is being shared with the bank. In case some new data is needed, the user is asked about it.
	\item{\textbf{Protection of Data :}}
	All the sensitive information about the user's personal identity resides with IMS. The bank's business information such as account information and policy also have to be stored at the IMS.
\end{enumerate}
\section{IDEMIX based solution}
Now we will evaluate our IDEMIX based solution as described in chapter 6. 
\begin{enumerate}
	\item{\textbf{Unlinkability :}}
	In the IDEMIX based system all the transactions are logged using a pseudonym from the IDEMIX token presented by the IMS. Hence, the user's real identity is unlinked from the transactions. If the same user logs in again,IMS creates a new token with a different pseudonym and its not possible to relate 2 sessions with each other.
	\item{\textbf{Escrow :}}
	As during creation of the presentation IDEMIX token, the real identity of the user is put in an inspectable field, so in case authorities need to get the real identity behind a pseudonym in transactions, they can just get it from the IDEMIX token from the bank.
	\item{\textbf{Minimal Technical Requirements :}}
	For the end users, there are not many technical requirements. They don't have to setup a special hardware to be a customer at the bank. All the interface is web based and can be used in any normal web browser. 
	\item{\textbf{User Awareness :}}
	IMS makes sure that the user knows what data about him is being shared with the bank in the presentation token. In case new data is needed the user is asked about it.
	\item{\textbf{Protection of Data}}
	All the sensitive information about the user's personal identity resides with IMS. Bank's business information such as account information and policy have to be stored in the IDEMIX credential at the  IMS. Only the bank is able to open this information as it is from a credential issued by the bank itself.
\end{enumerate}
\section{Discussion}
We discussed our solutions and evaluations with Nykredit and Signicat. Although it seems promising, the problem is that it is difficult to change the legacy systems. The IDEMIX solution seems better in fulfilling the goals of the system but it is still a new technology. To implement this system, a lot of data that Nykredit holds has to be moved out of the Nykredit systems. Even though that is the goal of the system, it seems like Nykredit is not ready to give up all the data out yet. Some of their internal business processes still rely heavily on the business data stored at their premises. 

Signicat, on the other hand, is really interested in the IDEMIX system. This will allow them to add one more service to their portfolio and strengthen their position in the market.

After all the discussion we reached the conclusion that even though the idea is revolutionary, maybe for Nykredit it is better to take one step at a time. Instead of completely anonymizing the system, they just want to replace user ID with a constant pseudonym. They want to keep all the policy and account database with them for the time being and want the IMS system to just replace user ID with a pseudonym for the banking purposes.

In this chapter we evaluated two different pseudonym systems as discussed in chapter 5 and 6. We then presented our discussion with the companies regarding these systems.
